
\vspace*{.5in}
\textbf{GADĪJUMLIELUMI}

\equationEntry{Sagaidāmā vērtība:}
{\begin{align*}
    E[X] &= \mu = \sum_{i=1}^{N}x_iP(X=x_i) \\
\end{align*}}

\equationEntry{Dispersija:}
{\begin{align*}
    \sigma^2 &= \sum_{i=1}^{N}[x_i-E[X]]^2P(X=x_i) = E[x^2] - (E[X])^2 \\
\end{align*}}

\equationEntry{Standartnovirze:}
{\begin{align*}
    \sigma &= \sqrt{\sum_{i=1}^{N}[x_i-E[X]]^2P(X=x_i)} = \sqrt{E[x^2] - (E[X])^2} \\
\end{align*}}

\equationEntry{Kovariācija:}
{\begin{align*}
     &= \sum_{i=1}^{N}[x_i-E[X]][y_i-E[Y]]P(X=x_i, Y=y_i)
\end{align*}}

\equationEntry{Divu diskrēto gadījumlielumu summa}
{\begin{align*}
    E(X+Y) = E(X) + E(Y)
\end{align*}}

\equationEntry{Divu diskrēto gadījumlielumu dispersija}
{\begin{align*}
    Var(X+Y) = \sigma_{x+y}^2 = \sigma{x} + \sigma{y} + 2\sigma_{xy}\\
\end{align*}}

\equationEntry{Divu diskrēto gadījumlielumu standartnovirze}
{\begin{align*}
    \sigma_{x+y} = \sqrt{\sigma_{x+y}} 
\end{align*}}


\equationEntry{Portfeļa ienesīgums (vidējais svērtais ienesīgums)}
{\begin{align*}
    E(P)=wE(X)+(1-w)E(Y)\\
\end{align*}

Kur $w$ = aktīva X īpatsvars portfelī $(1 - w)$ = aktīva Y īpatsvars portfelī.
}

\equationEntry{Portfeļa risks (svērtā izkliede)}
{\begin{align*}
    \sigma_P=\sqrt{w^2\sigma^2+(1-w)^2\sigma_Y^2+2w(1-w)\sigma_{XY}}
\end{align*}

Kur $w$ = aktīva X īpatsvars portfelī $(1 - w)$ = aktīva Y īpatsvars portfelī.
}

\equationEntry{Binomiālais sadalījums}
{\begin{align*}
    P(X=x | n,p)=\displaystyle\frac{n!}{x!(n-x)!}(1-\p)^{1-x}p^x
\end{align*}

Kur $p$ - notikuma varbūtība, $X$ - diskrēts gadījumlielums, $n$ - notikumu
(izmēģinājumu) skaits.}


\equationEntry{Binomiālais sadalījuma raksturlielumi}
{\begin{align*}
    \mu = E(X) &= np \\
    \sigma^2 &= np(1-p) \\
    \sigma &= \sqrt{np(1-p)}
\end{align*}}


\equationEntry{Puasona sadalījums (binom. aproksimācija)}
{\begin{align*}
    P(X=x|\lambda) &= \displaystyle\frac{e^{-\lambda}\lambda^x}{x!} \\
    P(X=x|\lambda) &= POISSON.DIST(X, \mu, CUMM.) 
\end{align*}

Kur $x$ = notikuma iestāšanās skaits, $\lambda$ = sagaidāmais notikumu skaits
(Puasona sadalījuma vidējā vērtība), $\lambda$ = np, $e$ = naturālā logaritma
bāze ($2.71828...$ vai $EXP(1)$)
}
\equationEntry{Puasona sadalījuma raksturlielumi}
{\begin{align*}
    \mu &= \lambda \\
    \sigma^2 &= \lambda \\
    \sigma &= \sqrt{\lambda}
\end{align*}
}

\equationEntry{Ģeometriskais sadalījums (Hiperģeometriskais)}
{\begin{align*}
    P(X=x|n,N,E)=\displaystyle\frac{(\frac{E}{x})(\frac{N-E}{n-x})}{(\frac{N}{n})} \\
\end{align*}

    Kur $N$ = ģenerālkopas lielums, $E$ = interesējošo vienumu skaits
    ģenerālkopā, $n$ = izlases lielums, $x$ = interesējošo vienumu skaits
    izlasē.
}


\equationEntry{Ģeometriskais sadalījuma raksturlielumi}
{\begin{align*}
    E[x]&=\displaystyle\frac{nE}{N} \\
    \sigma &= \sqrt{\displaystyle\frac{nE(N-E)}{N^2}}*\sqrt{\displaystyle\frac{N-n}{N-1}}
\end{align*}}




