\linebreak
\textbf{Notikumi}

\equationEntry{Kombinācijas (secība nav svarīga):}
{\begin{align*}
    C_n^k &= (\frac{n}{k}) = \frac{n!}{k!(n-k)!} = COMBIN(n, k)
\end{align*}}

\equationEntry{Kombinācijas ar atkārojumiem (secība nav svarīga):}
{\begin{align*}
    \overline{C}_n^k &= COMBINA(n, k) = TODO \\
\end{align*}}

\equationEntry{Variācijas(secība ir svarīga):}
{\begin{align*}
    A_n^k = \frac{n!}{(n-k)!} = PERMUT(n, k)
\end{align*}}

\equationEntry{Pretējais notikums:}
{\begin{align*}
    P(\lnot A ) = 1 - P (A)
\end{align*}}

\equationEntry{Neatkarīgo notikumu šķelums:}
{\begin{align*}
    P(A \land B) = P(A) * P(B)
\end{align*}}

\equationEntry{Neatkarīgo notikumu disjunkcija:}
{\begin{align*}
    P(A \lor B) = P(A) + P(B) - P(A \land B)
\end{align*}}

\equationEntry{Nesavienojamo notikumu disjunkcija:}
{\begin{align*}
    P(A \lor B) = P(A) + P(B)
\end{align*}}

\equationEntry{Nosacītā varbūtība:}

{

    $P(B|A)$ - B, ja zināms, ka A notika.
    \begin{align*}
        P(A|B) &= \displaystyle\frac{P(A \land B)}{P(B)} \\
        P(B|A) &= \displaystyle\frac{P(A \land B)}{P(A)}
    \end{align*}
}


\equationEntry{Nosacītā varbūtība (neatkarīgiem notikumiem):}

{

    $P(B|A)$ - B, ja zināms, ka A notika.
    \begin{align*}
        P(A|B) &= P(A) \\ 
        P(B|A) &= P(B) 
    \end{align*}
}


\equationEntry{Atkarīgo notikumu reizināšana:}

{

    $P(B|A)$ - B, ja zināms, ka A notika.
    \begin{align*}
        P(A) = P(A)P(B|A) \\
    \end{align*}
}


\equationEntry{Pilnā varbūtība}

{
    \displaystyle P(A)=\sum _{n}P(A\mid B_{n})P(B_{n})
}

\equationEntry{Beiesa teorēma}

{
    \displaystyle P(A\vert B)={\frac {P(B\vert A)P(A)}{P(B)}}

    Kur $A$ un $B$ ir notikumi un $P(B) \ne 0$. 
}
